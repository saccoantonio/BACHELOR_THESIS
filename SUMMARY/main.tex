\documentclass[12pt]{article}

% ======= LANGUAGE & ENCODING =======
\usepackage[utf8]{inputenc}
\usepackage[T1]{fontenc}
\usepackage[english]{babel}
\usepackage{microtype}

% ======= PAGE GEOMETRY & SPACING =======
\usepackage[a4paper, margin=2.5cm]{geometry}
\setlength{\parindent}{0pt}
\linespread{1.1}
\raggedbottom

\usepackage{enumitem}
\setlist[enumerate]{
    itemsep=1ex,
    topsep=0.5ex,
    parsep=0pt,
    partopsep=0pt,
    left=0.5em,
    labelsep=0.5em
}
\setlist[itemize]{
    itemsep=1ex,
    topsep=0.5ex,
    parsep=0pt,
    partopsep=0pt,
    left=0.5em,
    labelsep=0.5em
}

% ======= FONTS =======
\usepackage{lmodern}

% ======= MATHEMATICS =======
\usepackage{amsmath, amssymb, amsthm, mathtools}
\usepackage{bm}
\usepackage{physics}
\usepackage{siunitx}
\usepackage{cancel}
\usepackage{mathrsfs}
\usepackage{tikz-cd}
\usepackage{tensor}

% ======= GRAPHICS =======
\usepackage{graphicx}
\usepackage{float}
\usepackage{subcaption}
\usepackage{booktabs}
\usepackage{mdframed}
\usepackage{tikz}
\usetikzlibrary{calc, arrows.meta, decorations.pathmorphing}
\usepackage[table]{xcolor}

% ======= COLORS & LINKS =======
\usepackage{xcolor}
\definecolor{linkblue}{HTML}{1A4E8A}
\usepackage[
  colorlinks=true,
  linkcolor=linkblue,
  citecolor=linkblue,
  urlcolor=linkblue
]{hyperref}

% ======= CAPTIONS =======
\usepackage[font=small, labelfont=bf]{caption}

% ======= SECTION TITLES =======
\usepackage[explicit]{titlesec}
\titleformat{\section}[block]
  {\bfseries\LARGE}
  {\thesection}{1em}{#1}
\titleformat{\subsection}[block]
  {\bfseries\Large}
  {\thesubsection}{1em}{#1}


% ======= BIBLIOGRAPHY =======
\usepackage[backend=biber, style=numeric, sorting=none]{biblatex}
\addbibresource{chapters/bibliography.bib}

% ======= HEADERS & FOOTERS =======
\usepackage{fancyhdr}
\setlength{\headheight}{14pt}
\pagestyle{fancy}
\fancyhf{} % clear all fields

% === Gestione dei mark ===
\makeatletter
\renewcommand{\sectionmark}[1]{%
  \markright{Section \thesection\ — #1}%
}
\makeatother


% Numero di pagina al centro del footer
\fancyfoot[C]{\thepage}  

\renewcommand{\headrulewidth}{0.4pt}
\renewcommand{\footrulewidth}{0.4pt}
\newcommand \GB [1] {\bgroup\noindent[\textcolor{olive}{\textbf{GB}: #1}]\egroup\ignorespacesafterend}

% === Stile "plain" (usato all'inizio dei capitoli) ===
\fancypagestyle{plain}{
  \fancyhf{} % pulisce intestazioni e piè di pagina
  % Mantieni solo il fancyfoot come nel resto
  \fancyfoot[LE,RO]{%
    \tikz[baseline={(0,0)},anchor=center] \node[label={center:\thepage}]{};%
  }
  % Niente header (rimuove la linea in alto)
  \renewcommand{\headrulewidth}{0pt}
  % Mantieni la linea nel footer, se vuoi
  \renewcommand{\footrulewidth}{0.4pt}
}


% ======= FONT SIZE ADJUSTMENT =======
\makeatletter
\renewcommand\normalsize{%
   \@setfontsize\normalsize{13pt}{15.6pt}%
}
\makeatother
% -<0>-=v^v=--<0>-=^v^=-<0>-=v^v=-<0>-=^v^=-<0>-=v^v=-<0>-=^v^=-

\begin{document}

\begin{flushright}
\textbf{Antonio Sacco}  
\end{flushright}
\begin{center}
\textbf{\Large Thermal conductivity of Au systems: from bulk to nanoassembled films}
\end{center}

Understanding thermal transport in metallic nanosystems is crucial for designing nanoscale devices, where reduced size, morphological defects, including porosity, lack of periodicity strongly affect phonon propagation. While gold has well-known thermal conductivity in the bulk, Au-nanostructures exhibit more complex behaviour. 
This thesis investigates how thermal conductivity varies in Au systems varying their porosity and defect types, from bulk to nanopillars and deposited nanofilms.

We quantify the thermal transport employing equilibrium molecular dynamics (EMD) simulations performed with LAMMPS, coupled with the Green–Kubo (GK) formalism. Interactions between gold atoms are described using an embedded-atom method (EAM) potential. To carry out the full analysis, we develop a workflow composed of three main stages: (i) Atomic structures are generated using the ASE Python library. (ii) EMD simulations, as in LAMMPS, are run  to obtain the heat flux and the relevant dynamical information. (iii) Analysis of  the heat flux autocorrelation function (HFACF), computation of the Green–Kubo integrals, and extraction of structural descriptors such as the average coordination number (aCN) and the porosity (my Python code).

The main results of this thesis work are summarised in Figures \ref{fig:all_results} and \ref{fig:k_vs_cn}. These figures illustrate how various defect morphologies, including vacancies, voids, spherical holes formed within the bulk, and nanoparticle assemblies (pillars and films), impact the thermal conductivity of metallic systems.

\begin{figure}[H]
    \centering    
    \includegraphics[width=1\linewidth]{photos/all_results.jpg}
    \caption{Thermal conductivity as a function of porosity for the bulk, vacancies, holes, and void systems. All configurations display a monotonic decrease in conductivity with increasing porosity. The vacancies systems show the steepest reduction, followed by the holes and then the void systems. The nanofilm, in light blue, shows a conductivity similar to the last void systems, despite having three times the porosity. Darker points correspond to $\kappa_{FA}$, while lighter points represent $\kappa_{F}$, with their respective error bars.}
    \label{fig:all_results}
\end{figure}

Figure \ref{fig:all_results} compares the thermal conductivity of the different systems as a function of the porosity, excluding the nanopillar systems for which porosity is not well defined.
The results show a consistent trend across all configurations: thermal conductivity decreases monotonically with increasing porosity. Vacancy systems exhibit the strongest reduction if a uniform distribution of point defects occurs, while voids and hole-based structures preserve larger continuous regions that facilitate heat transport. 

When moving to the analysis of nanopillars, we observe that pillars carved from bulk crystals retain relatively high thermal conductivity, whereas pillars assembled from individual nanoparticles exhibit significantly lower values.

The deposited film studied in this work, despite its complex morphologies, returns conductivity values similar to those of systems with a single central void.

\begin{figure}[H]
    \centering
    \includegraphics[width=1\linewidth]{photos/k_vs_cn_fit.jpg}
    \caption{Thermal conductivity as a function of $\Delta \text{aCN}$, for all investigated systems. Two types of hole-containing systems are shown: (a) fixed radius with varying porosity, and (b) fixed porosity with varying hole radius. The dashed line represents an exponential fit $\kappa_{\mathrm{FA}} = a \, e^{-b \, (\Delta \text{aCN})} + c$ to the data, with fitted parameters $a = 5.329$, $b = 5.055$, and $c = 0.850$.
}
    \label{fig:k_vs_cn}
\end{figure}

In Figure \ref{fig:k_vs_cn}, thermal conductivity was analysed for all systems by plotting it against the coordination deficit $\Delta\text{aCN}=12-\text{aCN}$, which represents the deviation from the bulk value (aCN = 12). We found that thermal conductivity correlates with the mean coordination number. Low coordination, associated with an increased disorder and number of defects, implies a low conductivity. Conversely, higher coordination promotes better atomic connectivity and more efficient phonon transport.

Overall, this work demonstrates the robustness of GK-based thermal transport calculations in both ideal and highly disordered systems and provides insight into how low vs high mean coordination number governs the phononic contribution to heat conduction in gold.

\end{document}