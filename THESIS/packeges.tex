\usepackage[utf8]{inputenc}
\usepackage[T1]{fontenc}
\usepackage[english]{babel}
\usepackage{microtype}

\usepackage[a4paper, margin=2.5cm]{geometry}
\setlength{\parindent}{0pt}
\linespread{1.1}
\raggedbottom

\usepackage{enumitem}
\setlist[enumerate]{
    itemsep=1ex,
    topsep=0.5ex,
    parsep=0pt,
    partopsep=0pt,
    left=0.5em,
    labelsep=0.5em
}
\setlist[itemize]{
    itemsep=1ex,
    topsep=0.5ex,
    parsep=0pt,
    partopsep=0pt,
    left=0.5em,
    labelsep=0.5em
}

\usepackage{lmodern}

\usepackage{amsmath, amssymb, amsthm, mathtools}
\usepackage{bm}
\usepackage{physics}
\usepackage{siunitx}
\usepackage{cancel}
\usepackage{mathrsfs}
\usepackage{tikz-cd}
\usepackage{tensor}

\usepackage{graphicx}
\usepackage{float}
\usepackage{subcaption}
\usepackage{booktabs}
\usepackage{mdframed}
\usepackage{tikz}
\usetikzlibrary{calc, arrows.meta, decorations.pathmorphing}
\usepackage[table]{xcolor}

\usepackage{xcolor}
\definecolor{linkblue}{HTML}{1A4E8A}
\usepackage[
  colorlinks=true,
  linkcolor=linkblack,
  citecolor=linkblack,
  urlcolor=linkblack
]{hyperref}

\usepackage[font=small, labelfont=bf]{caption}

\usepackage[explicit]{titlesec}
\titleformat{\section}[block]
  {\bfseries\LARGE}
  {\thesection}{1em}{#1}
\titleformat{\subsection}[block]
  {\bfseries\Large}
  {\thesubsection}{1em}{#1}


\usepackage[backend=biber, style=numeric, sorting=none]{biblatex}
\addbibresource{chapters/bibliography.bib}

\usepackage{fancyhdr}
\setlength{\headheight}{14pt}
\pagestyle{fancy}
\fancyhf{} % clear all fields

\makeatletter
\renewcommand{\chaptermark}[1]{%
  \markboth{Chapter \thechapter\ — #1}{}%
}
\renewcommand{\sectionmark}[1]{%
  \markright{Section \thesection\ — #1}%
}
\makeatother

\fancyhead[LE]{\slshape\nouppercase{\leftmark}}  % Chapter a sinistra su pagine pari
\fancyhead[RO]{\slshape\nouppercase{\rightmark}} % Section a destra su pagine dispari
\fancyfoot[LE,RO]{%
  \tikz[baseline={(0,0)},anchor=center] \node[label={center:\thepage}]{};%
}

\renewcommand{\headrulewidth}{0.4pt}
\renewcommand{\footrulewidth}{0.4pt}
\newcommand \GB [1] {\bgroup\noindent[\textcolor{olive}{\textbf{GB}: #1}]\egroup\ignorespacesafterend}

\fancypagestyle{plain}{
  \fancyhf{} % pulisce intestazioni e piè di pagina
  % Mantieni solo il fancyfoot come nel resto
  \fancyfoot[LE,RO]{%
    \tikz[baseline={(0,0)},anchor=center] \node[label={center:\thepage}]{};%
  }
  % Niente header (rimuove la linea in alto)
  \renewcommand{\headrulewidth}{0pt}
  % Mantieni la linea nel footer, se vuoi
  \renewcommand{\footrulewidth}{0.4pt}
}

\makeatletter
\renewcommand\normalsize{%
   \@setfontsize\normalsize{13pt}{15.6pt}%
}
\makeatother